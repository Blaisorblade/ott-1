\mydummy{             Ott}
\mydummy{             }
\mydummy{by Peter Sewell and Francesco Zappa Nardelli.}


\subsection{Directory contents}
\mydummy{----------------}
The source distribution contains:\par\noindent
\begin{tabular}{ll}
\texttt{doc/}&                    the user guide, in html, pdf, and ps \\
\texttt{emacs/}&                  an Ott Emacs mode\\
\texttt{tex/}&                    auxiliary files for LaTeX\\
\texttt{coq/}&                    auxiliary files for Coq\\
\texttt{hol/}&                    auxiliary files for HOL\\
\texttt{tests/}&                  various small example Ott files\\
\texttt{examples/}&               some larger example Ott files\\
\texttt{src/}&                    the (OCaml) Ott sources\\
\texttt{bin/}&                    the Ott binary (binary distro only)\\
\texttt{Makefile} &               a Makefile for the examples\\
\texttt{LICENCE}&                 the BSD-style licence terms\\
\texttt{README}&                  this file (Section 2 of the user guide)\\
\texttt{revision\_history.txt}&                  the revision history\\
\texttt{ocamlgraph-0.99a.tar.gz}&  a copy of the \texttt{ocamlgraph} library
\end{tabular}\par\noindent
(we no longer provide a Windows binary distribution)
%The windows binary distribution omits the OCaml source files and
%\texttt{ocamlgraph} library, and adds a windows binary in the \texttt{bin/} directory.


\subsection{To build}
\mydummy{--------}
Ott depends on OCaml version 3.09.1 or later.  It builds with OCaml
3.12.1.  (Ott cannot be
compiled with OCaml 3.08, and it also touched an OCaml bug in 3.10.0 for amd64,
fixed in 3.10.1).



The command
\begin{alltt}
  make world
\end{alltt}
builds the \texttt{ott} binary in the \texttt{bin/} subdirectory.  

This will compiles Ott using \texttt{ocamlopt}.  To force it to
compile with \texttt{ocamlc} (which may give significantly slower execution
of Ott), do "\texttt{make world.byt}".


\subsection{To run}
\mydummy{------}
Ott runs as a command-line tool. Executing \texttt{bin/ott} shows the
usage and options.  To run Ott on the test file
\texttt{tests/test10.ott}, generating LaTeX in \texttt{test10.tex} and
Coq in \texttt{test10.v}, type:
\begin{alltt}
  bin/ott -i tests/test10.ott -o test10.tex -o test10.v
\end{alltt}
Isabelle and HOL can be generated with options \texttt{-o test10.thy} and
\texttt{-o test10Script.sml} respectively.

The Makefile has various sample targets, "\texttt{make tests/test10.out}",
"\texttt{make test7}", etc.  Typically they generate:\par\noindent
\begin{tabular}{ll}
  \texttt{out.tex}&        LaTeX source for a definition\\
  \texttt{out.ps}&         the postscript built from that\\
  \texttt{out.v}&          Coq source\\
  \texttt{outScript.sml}&  HOL source\\
  \texttt{out.thy}&        Isabelle source
\end{tabular}\par\noindent
from files \texttt{test10.ott}, \texttt{test8.ott}, etc., in \texttt{tests/}.


\subsection{Emacs mode}
\mydummy{----------}
The file \texttt{emacs/ottmode.el} defines a very simple Emacs mode for syntax
highlighting of Ott source files.  It can be used by, for example,
adding the following to your \texttt{.emacs}, replacing \texttt{PATH} by a path to your
Ott \texttt{emacs} directory.
\begin{alltt}
(setq load-path (cons (expand-file-name "PATH") load-path))
(require 'ottmode)
\end{alltt}


\subsection{Copyright information}
\mydummy{---------------------}
The \texttt{ocamlgraph} library is distributed under the LGPL (from
\ahrefurl{http://www.lri.fr/\home{filliatr/ftp/ocamlgraph/}}); we include a snapshot
for convenience. For its authorship and copyright information see the
files therein.

All other files are distributed under the BSD-style licence in \texttt{LICENCE}.

